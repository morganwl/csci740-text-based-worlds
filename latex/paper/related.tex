\section{Related work}

\subsection{Agents for playing text-based games}

Many AI agents have been written to attempt text-based games, some of
which were entered in the 2016, 2017 and 2018 \emph{Text-Based Adventure
AI Competition}. The first two competitions pitted agents against a
single game developed specifically for the competition, whereas the 2018
competition evaluated agents on their performance on 20 previously
released games\cite{atkinson_text-based_2019}. Agents were given 1000
moves per run of a game, repeated 10 times, and their average score,
measured as a fraction of the total possible score, was compared.  Three
agents have publicly available source code, BYUAgent
2016\cite{ricks_byu-agent-2016_2019}, Golovin\cite{kostka2017text} and
NAIL\cite{hausknecht_nail_2019}. This source code, along with published
work, provides some interesting approaches, and also elucidates some of
the thorny sub-problems that researchers are attempting to solve.

One particular challenge is identifying which verbs can be applied to
which objects, called \emph{affordance extraction}. \agent{BYUAgent
2016} uses word2vec word embeddings to create an \emph{affordance
vector} between objects extracted from game text and a pre-defined list
of verbs. Commands are generated from the most promising
verbs.\cite{atkinson_text-based_2019}
\agent{Golovin} learns command patterns by studying a collection of
published game solutions (called \emph{walkthroughs}), tutorials and
decompiled source code, and then uses a combination of LTSM neural
networks trained on fantasy novels and a word2vec embedding to match
those commands with objects extracted from scene
descriptions.\cite{kostka2017text}
\agent{NAIL} uses a 5-gram language model to estimate probable
affordance between extracted objects and a hand-curated list of 561
verbs mostly likely to occur in text-based
games.\cite{hausknecht_nail_2019}

Another challenge facing agents is building information about the
environment as it is explored, especially creating a map of locations
and pathways between them. Mapping, on its surface, seems simple enough
--- rooms are almost always reached by traveling in cardinal directions,
and these exits are usually (though not always) included in scene
descriptions. Unfortunately, multiple locations might bear the same
name, and the descriptions for locations can change over
time. \agent{Golovin} uses a heuristic for
identifying rooms and builds a simple room-direction graph. A MergeNode
algorithm is performed after every movement to merge nodes with
identical pathways to other nodes\cite{kostka2017text}. \agent{NAIL}
builds a much more detailed knowledge graph that includes not only rooms
and their exits, but also objects contained within those rooms. A fuzzy
string match is used to establish room
identity\cite{hausknecht_nail_2019}.

