\section{Related work}

\subsection{Frameworks for text-based games}

The excellent TextWorld paper lays out a formalization of text-based
games as Partially Observable Markov Decision Processes. The challenges
of text-based games in this context are explored, such as partial
observability, large state space, large and sparse action space,
exploration vs exploitation and long-term credit assignment. The
challenge of common sense reasoning and \emph{affordance extraction}, or
knowing which verbs are applicable to which objects, is explored.

The TextWorld framework aids in the study of text-based games by
providing a Python interface for connecting AI agents to any text-based
game published in the popular Z-machine format, and also provides tools
for procedurally generating text-based games with various constraints
and criteria\cite{cote_textworld_2019}.

\subsection{Agents for playing text-based games}

Many AI agents have been written to attempt text-based games, some of
which were entered in the 2016, 2017 and 2018 \emph{Text-Based Adventure
AI Competition}. The first two competitions pitted agents against a
single game developed specifically for the competition, whereas the 2018
competition evaluated agents on their performance on 20 previously
released games\cite{atkinson_text-based_2019}. Agents were given 1000
moves per run of a game, repeated 10 times, and their average score,
measured as a fraction of the total possible score, was compared.  Three
agents have publicly available source code, BYUAgent
2016\cite{ricks_byu-agent-2016_2019}, Golovin\cite{kostka2017text} and
NAIL\cite{hausknecht_nail_2019}. This source code, along with published
work, provides some interesting approaches, and also elucidates some of
the thorny sub-problems that researchers are attempting to solve.

NAIL, the winner of the 2018 competition, combines many components and a
number of specialized heuristics for text-based games. A 5-gram language
model is used to estimate probable affordance extractions. Information
about locations in the game, and all objects in it, is stored in a
knowledge graph, but not before being confirmed as \emph{valid} by a
custom validity detector. Room identity is determined with a fuzzy
string match. Decisions are made by a number of specialized decision
modules which take control of the agent based on their eagerness under
the current game state\cite{hausknecht_nail_2019}.

\agent{BYUAgent 2016} uses word2vec word embeddings to create an
\emph{affordance vector} between objects extracted from game text and a
pre-defined list of verbs. Commands are generated from the most
promising verbs.\cite{atkinson_text-based_2019}

\agent{Golovin} learns command patterns by studying a collection of
published game solutions (called \emph{walkthroughs}), tutorials and
decompiled source code, and then uses a combination of LTSM neural
networks trained on fantasy novels and a word2vec embedding to match
those commands with objects extracted from scene
descriptions. Locations are mapped using a simple room-direction graph.
A MergNode algorithm is performed after every movement to merge nodes
with identical pathways\cite{kostka2017text}.

Dambekodi et al have explored augmenting agents with the COMET inference
model and the BERT large language model, showing, in both cases, that
the statistical models encoded common sense which could be leveraged in
affordance extraction\cite{dambekodi_playing_2020}.

Haroush et al combined two deep learning algorithms to aid in action
selection. One learns the likelihood of action acceptance, while the
other learns a formal q-value.
