\section{Approach}

Because I am interested in processes that are interpretable, I chose to
focus on a knowledge-driven approach to solving text-based games. To
facilitate decision-making, I chose to use a combination of logical
reasoning and forward search. This makes the reasoning of the agent as
transparent as possible, allowing me to see what it is learning, what
conclusions it is making, and why it is making them.

\subsection{Core components}

In my approach, the agent can be seen as consisting of three fundamental
components. The \emph{Parser}, which interprets descriptions and
feedback from the game, the \emph{Knowledge Base}, which stores
observations made by the Parser and attempts to infer additional
knowledge, and the \emph{Decision Maker}, which uses information stored
in the Knowledge Base to find paths to goals and issue commands to the
game. In practice, the Decision Maker and the Knowledge Base are tightly
interwoven.

\begin{wrapfigure}{R}{.45\textwidth}
    \raggedright
    \texttt{
        >go west\\
        Forest\\
        This is a forest, with trees in all directions. To the east,
    there appears to be sunlight.}

    \caption{\small Game text from \game{Zork I}. The room
    name is on its own line, with no period. The word \emph{east} is
recognized by the agent as a potential exit.}
\end{wrapfigure}

The Parser needs to be able to receive text from the game and extract
from that new information about the game state, as well as information
about the outcome of actions. Because all game text is in the form of
natural language prose, this is not a simple process. This is a natural
problem for Natural Language Processing algorithms, but, because the
focus of this paper is on knowledge representation and inference, I
chose a simple keyword parser that extracts known object words from a
pre-determined list, using pre-determined information about those
objects to generate observations. For instance, a collection of
direction words can be represented as, \texttt{['north', 'south',
'east', \ldots,]}. Because my Parser was designed for movement only, I
was able to use a simple heuristic for recognizing a change in the game
state: after moving, the game prints a new room description, which
generally takes a recognizable form.

The Knowledge Base was my primary area of interest.
