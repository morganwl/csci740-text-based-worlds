\section{Description}

Text-based games, also called text adventures or interactive fiction,
are a variety of computer games originally appearing in 1976 with the
game \game{Adventure}\cite{crowther_adventure_1976}. The gameplay
experience is meant to evoke reading a work of fiction, where the
player controls a character, and through that character, the outcome of
the story. Early text-based games tended to be a mixture of puzzles and
exploration, with a simple goal, while more recent games have further
explored the potential of narrative control with diverging storylines
and interpretations. The games can be quite challenging and require
identifying and solving puzzles with common sense reasoning to achieve a
final goal. This challenge has recently made them an intriguing area of
research in Artificial Intelligence, though a generalized agent capable
of solving a wide range of text-based games remains
elusive\cite{cote_textworld_2019}.

\begin{wrapfigure}{R}{.45\textwidth}
    \small
    \raggedright
    \texttt{
        \noindent Alley\\
        This narrow aperture between two buildings is nearly blocked
        with piles of rotting cardboard boxes and overstuffed garbage
        cans. Ugly, half- crumbling brick walls to either side totter
        oppressively over you. The alley ends here at a tall, wooden
        fence.\\
        \medskip
        High up on the wall of the northern building there is a narrow,
        transom-style window.\\
        \medskip
        >examine window\\
        It's about eighteen inches wide and a foot tall, with hinges
        along the top that allow it to swing out. It's currently
        closed.\\
        \medskip
        >open window\\
        The window is too high.\\
        \medskip
        >move garbage can under window\\
        Grunting and holding your breath, you manhandle one of the
        filthy cans under the window.
    }
    \caption{Object interaction from
    \game{Anchorhead}\cite{gentry_anchorhead_1998}.  Player commands are
after the \texttt{>} prompt.}
\end{wrapfigure}

Players in text-based games control a \emph{player character} in an
environment which is rendered to the player entirely through prose
descriptions. The environment, which is unique from one game to the
next, generally models a lifelike world, governed by common sense rules.
The player character (which we will, by the genre's convention, simply
call \emph{player}), moves through discrete locations, or \emph{rooms}
in this world and interacts with objects contained within. All of these
actions are communicated to the game through natural language commands,
typed at a prompt. The commands are interpreted by the game's parser,
and the game responds, in natural language, based on the outcome of the
player's action.

While the prompt will accept any return-terminated string, the parser
understands only a small subset of English vocabulary, with simple
grammatical rules.\footnotemark Commands generally take the form of:
\texttt{verb [noun phrase [adverb phrase]]}, where \texttt{noun phrase}
often denotes an object in the environment and \texttt{adverb phrase}
often indicates a second object. Sample commands might be, ``\texttt{go
north}'', ``\texttt{look under bed}'' or ``\texttt{unlock wooden chest
with bronze key}.''\cite{cote_textworld_2019} All recognized words have
been explicitly coded, either by the authors of the parser, or by the
authors of the individual game, but these words are not necessarily
given to the player in an organized manner. Most games are written using
one of a few widely available frameworks, which include a parser with a
large vocabulary, but most games will make changes to that vocabulary or
the parser itself. The parser will reject commands that it does not
understand, though individual games might communicate this rejection in
different ways. The game will also often inform players of commands that
are understood but have no effect, or are deemed irrelevant to a game.

\footnotetext{Text-based games have been published in many natural
    languages, including Dutch, French, German, and Russian, but this paper
will focus exclusively on those available in English.} 

As a problem for an artificial intelligence agent, text-based games
present a number of clear challenges. The state space is finite, but
potentially large, and only partially observable. The transition model
is unknown, and the action-space is large, sparse and unknown. A verb
might only have meaning when applied to a single object in the game, and
it might only affect that object under specific conditions. Human
players navigate this action space through a mixture of common sense,
generalized and genre-specific knowledge, and ingenuity. Some games
require the use of made-up words. All information is given to the player
through natural language, from which meaningful state information and
action outcomes must be extracted. Some information is given to the
character unbidden, but other information must be asked explicitly by
examining objects in the environment. Furthermore, each game has its own
unique goal (or goals), which may or may not be clearly communicated to
the player.

The challenges of text-based games also make them an interesting
environment for developing AI agents. An AI agent must be capable of
extracting information from natural language processing, recognizing
objects of interest, changes in the game environment, and other
potential clues. The agent must be able to employ, or simulate, common
sense reasoning, recognizing which actions might impact which actions,
and combining those actions to reach goals.

While generalized text-based games remain unsolved by AI agents, agents
have achieved partial successes in those environments, and have been
able to solve constrained games created as benchmarks. This project aims
to create an AI agent capable of solving games that have been
constrained in the goal and the kind of puzzles that must be solved to
achieve it. Namely, the agent will be asked to solve mazes broken into
four difficulty levels.

\begin{enumerate}
    \item Trivial mazes (interconnected rooms without obstacles)
    \item Familiar obstacles requiring one action, such as doors
    \item Composite obstacles requiring multiple actions, such as locked
        doors
    \item Complex, novel obstacles, such as laying a plank across a
        chasm
\end{enumerate}

The agent's performance will be compared to existing agents, and,
hopefully, it will exceed them under some if not all circumstances.
