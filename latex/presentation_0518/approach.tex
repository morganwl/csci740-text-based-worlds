\section{Approach}

\begin{frame}
    \frametitle{Approach}
    \framesubtitle{Components}

    \small

    \begin{block}{Parser}
        Read text feedback from game.

        Distinguish command acceptance from command
        rejection.

        Extract observations and \emph{tell} them to
        Knowledge Base.

        \textbf{Natural Language Processing}
    \end{block}

    \begin{block}{Knowledge Base}
        Attempt to infer new facts from observations.

        Distinguish known from unknown.


        Provide exploration goals.

        \textbf{Logical reasoning with forward chaining}
    \end{block}

    \begin{block}{Decision Maker}
        Prioritize goals.

        Find paths to goal states.

        Generate commands.

        \textbf{Depth-based search}
    \end{block}

    % \begin{block}{NLP-assisted groupings in hierarchical knowledge
    %     base}
    %     \begin{itemize}
    %         \item Crude parsing of scene descriptions using word
    %             embeddings
    %         \item Use knowledge base to encode observations of objects
    %             in rooms
    %         \item Using word embeddings, navigate a hierarchical
    %             knowledge base tying observations of verbs applied to
    %             objects to look for matches
    %         \item Exploit strong matches or continue exploration
    %     \end{itemize}
    % \end{block}

\end{frame}

\begin{frame}
    \frametitle{Linear first-order logic}
    \framesubtitle{The underpinnings of the Knowledge Base}

    \begin{itemize}
        \item An extension of first-order logic where \emph{linear
            implication} rules \emph{consume} (i.e. negate) the
            conditions in the premise as well as asserting the
            consequent.

        \item Linear implication allows \emph{actions} to change the
            state of facts in the knowledge base.

        \item Facts are stored as predicates on \emph{constants} (i.e.
            objects).
    \end{itemize}

    \begin{figure}
        {\tiny
            \[
                go(DIR) :: at(player, L) \otimes
                \$ connects(L, DIR, DEST) \otimes
                \lnot \$ blocked(L, DEST)
                \multimap at(player, DEST) 
            \]
        }
        \caption*{Linear implication rule for travelling between rooms}
    \end{figure}

    \parencite{cote_textworld_2019}

\end{frame}

\begin{frame}
    \frametitle{Forward chaining}
    \framesubtitle{Inference with first-order logic}

    Whenever new facts are \emph{told} to the knowledge base, these
    facts might imply additional facts.

    \begin{enumerate}
        \item Iterate over known rules
            \begin{enumerate}
                \item Find all substitutions for the premise that are
                    entailed
                \item For each valid substitution, see if the consequent
                    unifies with a fact in the database
                \item If not, store the consequent and mark
                    \emph{changed}
            \end{enumerate}
        \item If \emph{changed} is True, return to 1
    \end{enumerate}

    \begin{figure}
        \emph{Unification} is the algorithm by which variable substitutions
        are found which make two logical sentences equal.

        \begin{gather*}
            connects(L, DIR, DEST) = connects(kitchen, west, hallway) \\
            \text{where } \lbrace{L: kitchen, DIR: west, DEST:
            hallway\rbrace}
        \end{gather*}
    \end{figure}
\end{frame}

